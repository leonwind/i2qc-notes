\section{The Density Operator}
So far: State vector $\statepsi$ describing a quantum state.
Convenient alternative formulation for quantum systems about which we 
only have partial information: \\
\underline{Density operator} (also called density matrix)

\subsection{Ensembles of quantum states}
Consider a quantum system which is in one of several states $\ket{\psi_i}$
with probability $p_i$: \underline{Ensemble} of quantum states
$\{p_i, \ket{\psi_i} \}$ \\
The \underline{density operator} $\rho$ of the ensemble $\{p_i, \ket{\psi_i} \}$
is defined as 

\begin{equation}
    \rho = \sum_i p_i \ket{\psi_i} \bra{\psi_i}
\end{equation}

Quantum mechanics in terms of the density operators:
\begin{itemize}
    \item Unitary operation: a unitary transformation $U$ maps each
    $\ket{\psi_i} \mapsto U \ket{\psi_i}$, and the ensemble to  $\{p_i, U \ket{\psi_i} \}$.
    Thus the density operator is transformed as 
    \begin{equation}
        \rho \stackrel{U}{\mapsto} \sum_i p_i U \ket{\psi_i} 
            \underbrace{\bra{\psi_i} U^\dag}_{(U\ket{\psi_i})^\dag} 
        = U \underbrace{(\sum_i p_i \ket{\psi_i} \bra{\psi_i})}_{\rho} U^\dag 
        = U \rho U^\dag
    \end{equation}

    \item Measurements: measurement operators $\{M_m\}$, if the system is in state
    $\ket{\psi_i}$, then the probability for result $m$, given $i$, is 
    \begin{equation}
        p(m|i) = \braketmatrix{\psi_i}{M_m^\dag M_m}{\psi_i} = 
            tr \left[ M_m^\dag M_m \ket{\psi_i} \bra{\psi_i} \right]
    \end{equation}
    Thus overall probability for result m is:
    \begin{align}
        p(m) &= \sum_i p(m|i) p_i = 
            \sum_i tr \left[ M_m^\dag M_m \ket{\psi_i} \bra{\psi_i} \right] p_i \\
            &= tr [ M_m^\dag M_m \underbrace{\sum_i p_i \ket{\psi_i} \bra{\psi_i}}_{\rho}] 
            = tr [M_m^\dag M_m \rho]
    \end{align}

    Density operator $\rho_m$ after obtaining measurement result $m$? \\
    State $i$ collapses to 
    \begin{equation}
    \ket{\psi_i} \mapsto \frac{M_m \ket{\psi_i}}{|| M_m \ket{\psi_i}||} =: \ket{\psi_i^m}
    \end{equation}
    Thus:
    
    \begin{align}
        \rho_m &= \sum_i p(i|m) \ket{\psi_i^m} \bra{\psi_i^m} 
            = \sum_i p(i|m) \frac{M_m \ket{\psi_i} \bra{\psi_i^m} M_m^\dag}
            {\underbrace{|| M_m \ket{\psi_i}||^2}_{p(m|i)}} \\
            &\stackrel{\text{Bayes Theorem}}{=} \sum_i p_i \frac{M_m \ket{\psi_i} \bra{\psi_i} M_m}{p(m)}
            = \frac{M_m \rho M_m^\dag}{tr[M_m^\dag M_m \rho]}
    \end{align}

    Note that $\rho_m$ is now expressed in terms of $\rho$ and the measurement
    operators, without explicit reference to the ensemble $\{p_i, \ket{\psi_i} \}$.
\end{itemize}

\subsection{General properties of the density operator}

Characterization of density operators: An operator $\rho$ is the
density matrix associated to some ensemble $\{p_i, \ket{\psi_i} \}$ if and only if:
\begin{enumerate}
    \item $tr[\rho] = 1$ (\textit{trace condition})
    \item $\rho$ is a positive operator (\textit{positivity condition})
\end{enumerate}
Remark: $\rho$ is called a \underline{positive operator} if it is Hermitian and all its
eigenvalues are non-negative, equivalent 
$\braketmatrix{\varphi}{\rho}{\varphi} \geq 0$ for all vectors $\ket{\varphi}$. 

Proof is in the official lecture notes on Moodle. \\

From now on, we define a density operator as positive operator $\rho$ with
$tr[\rho] = 1$. \\
Language regarding density operators: \\
\begin{enumerate}
    \item \underline{"Pure state"}: Quantum system in a state $\statepsi$, corresponding density 
operator $\rho = \statepsi \bra{\psi}$ such that 
\begin{equation}
    tr[\rho^2] = tr[\ket{\psi} \underbrace{\braket{\psi}{\psi}}_{1} \bra{\psi}] 
    = \braket{\psi}{\psi} = 1
\end{equation}

    \item \underline{"Mixed State"}: $\rho$ describing quantum setup cannot be written 
    as $\rho = \statepsi \bra{\psi}$; Intutition: Ensemble $\{p_i, \ket{\psi_i} \}$ of 
    $\rho$, all the probabilities are strictly smaller than 1. 
    Then $tr[\rho^2] = \sum_i p_i^2 < 1$.
\end{enumerate}

In general: Let $\rho$ be a density operator. Then  $tr[\rho^2] \leq 1$, 
and $tr[\rho^2] = 1$ if and only if $\rho$ describes a pure quantum state. \\
Proof: Denote the eigenvalues of $\rho$ by $\{\lambda_i \}$, then $- \leq \lambda_i \leq 1$
since $\rho$ is positive and $1 = tr[\rho] = \sum_i \lambda_i$.
Moreover, $tr[\rho^2] = \sum_i \lambda_i^2 \leq 1$, with "$=1$" precisely if one of the 
eigenvalues is 1 and the others are 0. \\


Ensemble representation is not unique! Example:
\begin{equation*}
    \rho = \frac{3}{4} \ket{0} \bra{0} + \frac{1}{4} \ket{1}\bra{1} 
    = \frac{1}{2} \ket{a} \bra{a} + \frac{1}{2} \ket{b}\bra{b} = 
\end{equation*}
with 
\begin{align*}
    \ket{a} &= \sqrt{\frac{3}{4}} \ket{0} + \sqrt{\frac{1}{4}} \ket{1} \\
    \ket{b} &= \sqrt{\frac{3}{4}} \ket{0} - \sqrt{\frac{1}{4}} \ket{1} \\
\end{align*}

(But note that $\ket{0}, \ket{1}$ are the (unique) eigenvectors of $\rho$, 
and $\braket{a}{b} \neq 0$.)